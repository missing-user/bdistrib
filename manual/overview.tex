\chapter{Overview}

This program computes efficiency-ordered distributions of magnetic fields as described in \cite{boozer2015}.  The main method is described on page 12, and some key definitions are given on page 9.

There are 3 surfaces, named 'plasma', 'middle', and 'outer', The 'plasma' surface can correspond to the outermost surface of a VMEC equilibrium, or it can be a plain circular toroidal surface. The name of the relevant surface appears as a suffix on
most variables.

On each surface we use a poloidal angle u and a toroidal angle v, defined as in NESCOIL.  The coordinate u lies in the range [0,1]. The stellarator has nfp identical toroidal periods, and an increase in v by 1 corresponds to 1 of these periods. Thus,
v increases by nfp in a complete toroidal revolution. In the output file, there is an array v corresponding to one toroidal period, as well as an array vl corresponding to all nfp toroidal periods.  Note that v is proportional to the standard
cylindrical angle phi: phi = 2*pi*v/nfp.  Generally, field lines are not straight in the (u,v) coordinates.  On the plasma surface, u is identical to the VMEC poloidal angle divided by 2*pi, while the VMEC toroidal angle differs by a sign relative to
v.

In the various variable names in the code and output file, 'r' refers to the position vector, not to a radius.  In various arrays with a dimension of length 3, this dimension always corresponds to Cartesian coordinates (x,y,z).

The 'normal' quantities in the code and output file refer to the surface normal vector N = (dr/dv) cross (dr/du) as in the NESCOIL paper. Note that this vector does not have unit magnitude.



\section{Required libraries}

\begin{itemize}

\item {\ttfamily LIBSTELL} (for reading \vmec~{\ttfamily wout} files, and for the {\ttfamily ezcdf} subroutines)
\item {\ttfamily NetCDF} (for writing the output file)
\item {\ttfamily LAPACK} (for the SVD subroutine)

\end{itemize}

If OpenMP is available, calculations with the code are parallelized.  The plotting and testing functions use \python,
{\ttfamily numpy}, and {\ttfamily scipy}.
The plotting routine \bdistribPlot~uses {\ttfamily matplotlib}.

\section{Questions, Bugs, and Feedback}

We welcome any contributions to the code or documentation.
For write permission to the repository, or to report any bugs, provide feedback, or ask questions, contact Matt Landreman at
\href{mailto:matt.landreman@gmail.com}{\nolinkurl{matt.landreman@gmail.com} }






